\documentclass[11pt, oneside]{article}   	% use "amsart" instead of "article" for AMSLaTeX format
\usepackage{geometry}                		% See geometry.pdf to learn the layout options. There are lots.
\geometry{letterpaper}                   		% ... or a4paper or a5paper or ... 
%\geometry{landscape}                		% Activate for for rotated page geometry
%\usepackage[parfill]{parskip}    		% Activate to begin paragraphs with an empty line rather than an indent
\usepackage{graphicx}				% Use pdf, png, jpg, or eps� with pdflatex; use eps in DVI mode
								% TeX will automatically convert eps --> pdf in pdflatex		
\usepackage{amssymb}


\usepackage{xcolor,cancel}

\newcommand\hcancel[2][black]{\setbox0=\hbox{$#2$}%
\rlap{\raisebox{.45\ht0}{\textcolor{#1}{\rule{\wd0}{1pt}}}}#2} 

\newcommand{\vect}[1]{\mathbf{#1}}

\title{Assignment 1}
\author{Arvid Frydenlund}
%\date{}							% Activate to display a given date or no date

\begin{document}
\maketitle
\section{Q1}

softmax(${\vect x}$) = softmax(${\vect x}+c$)  (Note, remember softmax is defined as a function taking in a vector)

Considering element $i$

\begin{equation}
\frac{e^{x_i}}{\sum_j e^{x_j}} = \frac{e^{x_i+ c}}{\sum_j e^{x_j + c}}
\end{equation}

since,

\begin{equation}
\frac{e^{x_i}e^c}{\sum_j e^{x_j}e^c} = \frac{e^{x_i} \hcancel[red]{e^c}}{\sum_j e^{x_j} \hcancel[red]{e^c} } 
\end{equation}

QED


This is useful for the trick where $c =  -\rm{max^n_i} \vect{x}_i$, since the largest number in the exponential possible is now 0.  That is it prevents overflow.   


\section{Q2 a)}


The sigmoid function is 

\begin{equation}
y = \frac{1}{1+e^{-x}} = \sigma(x) = (1+e^{-x})^{-1} = (z)^{-1}
\end{equation}

Where $z = 1+e^{-x} = 1 + e^r$, and $r = -x$

Then by simple chain rule

\begin{equation}
\frac{dy}{dx} = \frac{dz}{dz}*\frac{dz}{dr}*\frac{dr}{dx} 
\end{equation}

\begin{equation}
\frac{dy}{dx} = -(1+e^{-x} )^{-2} * e^{-x} * (-1) = e^{-x} * \bigg(\frac{1}{1+e^{-x}} \bigg)^2 = e^{-x}  \sigma(x) \sigma(x)
\end{equation}


Since $e^{-x} = \frac{1-\sigma(x)}{\sigma(x)}$

because 

\begin{equation}
\frac{1 - \frac{1}{1+e^{-x}}}{\frac{1}{1+e^{-x}}} =(1+e^{-x}) -\frac{1+e^{-x}}{1+e^{-x}} = (1+e^{-x}) -1 = e^{-x}
\end{equation}

Then

\begin{equation}
e^{-x}  \sigma(x) \sigma(x) = \frac{1-\sigma(x)}{\sigma(x)} \sigma(x) \sigma(x)   = (1 - \sigma(x) ) (\sigma(x) )
\end{equation}

QED



\section{Q2 b)}


\end{document}  